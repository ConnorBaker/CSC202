% Author: Connor Baker
% Date Created: November 3, 2016
% Last Edited: November 5, 2016
% Version: 0.3a

\documentclass[12pt]{article}
\usepackage[utf8]{inputenc}
\usepackage{amsmath}
\usepackage{enumitem}
\usepackage[left=1.5in,right=1.5in,top=1.5in,bottom=1.5in]{geometry}

\begin{document}

% Creates a title page with content centered vertically/horizontally
\null
\nointerlineskip 
\vfill
\let \snewpage \newpage
\let \newpage \relax
    \title{Documentation: assignment6}
    \author{Connor Baker}
    \date{November 7, 2016\\Version 0.5a}
\maketitle
\let \newpage \snewpage
\vfill



\newpage % New page



%Creates a table of contents with dotted lines
\makeatletter
\renewcommand*\l@section{\@dottedtocline{1}{0em}{1.5em}}
\makeatother

\tableofcontents
\clearpage

\begin{center}
\section{Summary of Problem Specification}
\end{center}
\subsection{Abstract}
Using a standard tuning frequency inputted by the user, as well as a range of values for capacitance, we calculate possible frequencies greater than the user inputted standard tuning frequency, writing the values to a random access, byte-based file.
\par % New paragraph
We begin by grabbing user input, and then calculate capacitance using the formulae given in Section 2.We then find the inductance (\textit{L}) using the standard tuning frequency. After that, we calculate our possible range of values for frequency (\textit{F\textsubscript{min}} and \textit{F\textsubscript{max}}), comparing each to the standard tuning frequency, writing only those that are greater to file. At this point, we calculate new values of \textit{F\textsubscript{max}} by incrementing \textit{C\textsubscript{min}} by 15 picofarad, again comparing and writing only values larger than the standard tuning frequency to file. We do this until \textit{C\textsubscript{min}} surpasses \textit{C\textsubscript{max}}. When this happens, we increment \textit{L} by 2\%, reset \textit{C\textsubscript{min}}, and again calculate values of \textit{F\textsubscript{min}} and \textit{F\textsubscript{max}} using the same process described above.
\par % New paragraph
Each time we edit or do not edit the file, print out to console why. For example, if our \textit{F\textsubscript{max}} is larger than 16.7MHZ and we edit the file to include that new value, we should print to console that we have done so.
\par % New paragraph
The program halts when the largest value of \textit{F\textsubscript{max}} is smaller than the standard tuning frequency. It should be noted that (conveniently) the first calculated value of \textit{F\textsubscript{max}} is the greatest (for our purposes at least), as it is strictly monotonic decreasing in nature, given a strictly monotonic increasing \textit{L} and \textit{C\textsubscript{min}}, which is exactly what we have.
\subsection{Assumptions}
I use a pre-release version of Java 9. It is my assumption that the underlying changes in the language were nothing such that it would allow me to write something incompatible with the immediate previous release.



\newpage % New page



\begin{center}
\section{Formulae}
\end{center}
Capacitance is denoted with \textit{C} and measured in farads.
Capacitance Minimum is denoted with \textit{C\textsubscript{min}}.
Capacitance Minimum is denoted with \textit{C\textsubscript{max}}.
Frequency is denoted with \textit{F} and measured in hertz.
Frequency Minimum is denoted with \textit{F\textsubscript{min}}.
Frequency Minimum is denoted with \textit{F\textsubscript{max}}.
Inductance is denoted with \textit{L} and measured in henrys.
\par % New paragraph
\begin{equation}
\textit{C} = \sqrt{\textit{C\textsubscript{min}}*\textit{C\textsubscript{max}}}
\end{equation}
\begin{equation}
\textit{F\textsubscript{min}} = \frac{2\pi}{\sqrt{\textit{L}*\textit{C\textsubscript{max}}}}
\end{equation}
\begin{equation}
\textit{F\textsubscript{max}} = \frac{2\pi}{\sqrt{\textit{L}*\textit{C\textsubscript{min}}}}
\end{equation}
\begin{equation}
\textit{L} = \frac{(\frac{2\pi}{\textit{F}})^2}{\textit{C}}
\end{equation}



\newpage % New page



\begin{center}
\section{Explanation of Components}
\end{center}

\subsection{Main.class}
Main.class contains the main() method.
\subsubsection{The main() method}
The main() method consists of three main parts: the body, outer while loop, and inner while loop.

\subparagraph{The body}
\begin{itemize}[leftmargin=1.5em]
\item[] The body of the method creates initializes the object we operate on (a component of TuningCircuit.class and covered in the corresponding subsection), as well as the random access byte-based file the program writes all the doubles to.
\end{itemize}

\subparagraph{The outer while loop}
\begin{itemize}[leftmargin=1.5em]
\item[] The outer while loop has three duties:
\begin{enumerate}
\item To check for a condition that signals the termination of the program 
\item To perform multiple iterations over the inner while loop
\item To increment the value of \textit{L} and reset \textit{C\textsubscript{min}} to the initial value input by the user every time the inner while loop terminates.
\end{enumerate}
\item[] The first duty is met via a simple if() statement. If the value of \textit{F\textsubscript{max}} is less than the user inputted standard tuning frequency, we can halt calculations because every value calculated after that point for the current or larger \textit{L} and any given \textit{C\textsubscript{min}} will be smaller as well.
\item[] The second duty is met as this component of the main() method is a while loop.
\item[] The third duty is met by performing operations on the object using methods found in TuningCircuit.class.
\end{itemize}

\subparagraph{The inner while loop}
\begin{itemize}[leftmargin=1.5em]
\item[] The inner while loop has three duties:
\begin{enumerate}    
\item To calculate values of frequencies larger than the standard tuning frequency inputted by the user given a certain capacitance and inductance and print to file/console
\item To increment \textit{C\textsubscript{min}} by 15 picofarad
\item To check for a condition that signals us to break out of the inner while loop
\end{enumerate}
\item[] The first duty is met by calling methods located in TuningCircuit.class.
\item[] The second duty is met by using the compound assignment operator +=.
\item[] The third duty is met in the form of a if() statement that checks for values of  \textit{F\textsubscript{max}} that are smaller than the standard tuning frequency. When this occurs, we exit the inner while loop.
\end{itemize}


\subsection{TuningCircuit.class}
\subsubsection{The TuningCircuit() constructor}
The TuningCircuit() constructor calls several methods to get values for the variables used in the program. It calls (in order):
\begin{enumerate}
\item promptUserForStandardFrequency()
\item promptUserForCapacitanceMinimum()
\item promptUserForCapacitanceMaximum()
\item calculateCapacitance()
\item calculateInductance()
\item calculatefMin()
\item calculatefMax()
\end{enumerate}
These methods are detailed below.

\subsubsection{The promptUserForStandardFrequency() method}
Prints a request for the user to input the standard frequency in hertz. Grabs input via InputStreamReader wrapped in BufferedReader. Accepts doubles.
\subsubsection{The promptUserForCapacitanceMinimum() method}
Prints a request for the user to input the lower bound of capacitance in farad. Grabs input via InputStreamReader wrapped in BufferedReader. Accepts doubles.
\subsubsection{The promptUserForCapacitanceMaximum() method}
Prints a request for the user to input the upper bound of capacitance in farad. Grabs input via InputStreamReader wrapped in BufferedReader. Accepts doubles.
\subsubsection{The calculateCapacitance() method}
Calculates capacitance by using equation (1) from Section 2 (Formulae).
\subsubsection{The calculateInductance() method}
Calculates inductance by using equation (4) from Section 2 (Formulae).
\subsubsection{The calculatefMin() method}
Calculates the lower bound of freqency by using equation (2) from Section 2 (Formulae).
\subsubsection{The calculatefMax() method}
Calculates the upper bound of freqency by using equation (3) from Section 2 (Formulae).



\newpage % New page



\begin{center}
\section{Notes}
\end{center}
Equations (1--4) relate everything in terms of base units. As such, we ask that the user input all required variables in terms of their respective base units.

\subsection{Todo}
This program isn't as fast as I would like it to be. Ideally, I could find a way to calculate the number of iterations that the current while loops would perform for any given standard tuning frequency and range of capacitance. From there, I could attempt to extract parallelism in the form of incrementing by staggered amounts and creating multiple threads. This would allows the program to finish much more quickly, or, could be done to ensure more reasonable run time if we should choose to expose the amount we increment \textit{C\textsubscript{min}} by (thus allowing for smaller values and more possible frequencies). However, as we currently write to file, I'm not sure how I could implement such a task in a thread-safe way.



\newpage % New page



\begin{center}
\section{References}
\end{center}
http://download.java.net/java/jdk9/docs/api/

\end{document}