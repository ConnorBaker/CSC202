% Author: Connor Baker
% Date Created: November 15, 2016
% Last Edited: November 15, 2016
% Version: 0.1a

\documentclass[12pt]{article}
\usepackage[utf8]{inputenc}
\usepackage{amsmath}
\usepackage{enumitem}
\usepackage[left=1.5in,right=1.5in,top=1.5in,bottom=1.5in]{geometry}

\begin{document}
% Creates a title page with content centered vertically/horizontally
\null
\nointerlineskip 
\vfill
\let \snewpage \newpage
\let \newpage \relax
    \title{assignment9}
    \author{Connor Baker}
    \date{November 15, 2016\\Version 0.1a}
\maketitle
\let \newpage \snewpage
\vfill
\thispagestyle{empty}



\newpage % New page



% Create the table of contents formatted with dotted lines and roman page numbering
\makeatletter
\pagenumbering{roman}
\renewcommand*\l@section{\@dottedtocline{1}{0em}{1.5em}}
\makeatother
\tableofcontents

% End the table of contents and reset page numbering to arabic
\clearpage
\pagenumbering{arabic}

% Create the Summary of Problem Specification section
\begin{center}
\section{Summary of Problem Specification}
\end{center}
\subsection{Abstract}
Using the inverse of a matrix find the solution to a system of linear equations.
\subsection{Algorithm}
Though very slow, the algorithm that we'll use for finding the inverse involves the use of the determinant and the adjugate (and thus cofactor) matrices. This kind of algorithm is best expressed through an example:\\
Consider the system:\\
\begin{equation*}
    x_1 + 3x_2 - 2x_3 = 5
\end{equation*}
\begin{equation*}
    2x_1 + 5x_3 = 10
\end{equation*}
\begin{equation*}
    x_1 - x_2 - 10x_3 = 2
\end{equation*}
We can represent this linear system of equations as an augmented matrix:
\[
A =
\left[ 
\begin{array}{ccc|c}
    1 & 3 & -2 & 5 \\
    2 & 0 & 5 & 10 \\
    1 & -1 & -10 & 2
\end{array}
\right]
\]
\begin{enumerate}
\item Look only at the coefficient matrix, and begin to calculate the cofactor matrix. Let our coefficient matrix be $C$. Then:
\[
C =
\begin{bmatrix}
    1 & 3 & -2 \\
    2 & 0 & 5 \\
    1 & -1 & -10
\end{bmatrix}
\]
And our cofactor matrix $C'$ will be the result of:
\[
C' =
\begin{bmatrix}
    % First row of cofactor matrix
    +\begin{vmatrix}
    0 & 5\\
    -1 & -10
    \end{vmatrix} & 
    -\begin{vmatrix}
    2 & 5\\
    1 & -10
    \end{vmatrix} & 
    +\begin{vmatrix}
    2 & 0\\
    1 & -1
    \end{vmatrix}\\\\
    
    % Second row of cofactor matrix
    -\begin{vmatrix}
    3 & -2\\
    -1 & -10
    \end{vmatrix} & 
    +\begin{vmatrix}
    1 & -2\\
    1 & -10
    \end{vmatrix} & 
    -\begin{vmatrix}
    1 & 3\\
    1 & -1
    \end{vmatrix}\\\\
    
    % Third row of cofactor matrix
    +\begin{vmatrix}
    3 & -2\\
    0 & 5
    \end{vmatrix} & 
    -\begin{vmatrix}
    1 & -2\\
    2 & 5
    \end{vmatrix} & 
    +\begin{vmatrix}
    1 & 3\\
    2 & 0
    \end{vmatrix}
\end{bmatrix}
\]
\item Evaluating $C'$ yields the matrix:
\[
C' =
\begin{bmatrix}
    5 & 25 & -2 \\
    32 & -8 & 4 \\
    15 & -9 & -6
\end{bmatrix}
\]
\item We then take the transpose of $C'$, called the adjugate of $C$:
\[
adj(C) =
\begin{bmatrix}
    5 & 32 & 15 \\
    25 & -8 & -9 \\
    -2 & 4 & -6
\end{bmatrix}
\]
\item Now that we've found the adjugate of $C$, we simply multiply that matrix by the reciprocal of the determinant to find the inverse of $C$.
\[
C^{-1} = \frac{1}{det(C)} * adj(C) =
\begin{bmatrix}
    5/69 & 32/69 & 15/69 \\
    25/69 & -8/69 & -3/23 \\
    -2/69 & 4/69 & -2/23
\end{bmatrix}
\]
\item Finally, we need only multiply the inverse by the solutions vector to find values of $x$.
\[
Cx = b
\]
\[
\begin{bmatrix}
    1 & 3 & -2 \\
    2 & 0 & 5 \\
    1 & -1 & -10
\end{bmatrix}
\begin{bmatrix}
    x_1 \\
    x_2 \\
    x_3
\end{bmatrix}
=
\begin{bmatrix}
    5 \\
    10 \\
    2
\end{bmatrix}
\]
\[
C'Cx = C'b
\]
\[
Ix = C'b
\]
\[
x = C'b
\]
\[
\begin{bmatrix}
    x_1 \\
    x_2 \\
    x_3
\end{bmatrix}
=
\begin{bmatrix}
    5 \\
    10 \\
    2
\end{bmatrix}
\begin{bmatrix}
    5/69 & 32/69 & 15/69 \\
    25/69 & -8/69 & -3/23 \\
    -2/69 & 4/69 & -2/23
\end{bmatrix}
\]
\[
\begin{bmatrix}
    x_1 \\
    x_2 \\
    x_3
\end{bmatrix}
=
\begin{bmatrix}
    125/23 \\
    9/23 \\
    6/23
\end{bmatrix}
\]
\end{enumerate}
\end{document}
